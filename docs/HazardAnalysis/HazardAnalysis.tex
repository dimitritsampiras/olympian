

\documentclass{article}

\usepackage{booktabs}
\usepackage{tabularx}
\usepackage{hyperref}

\hypersetup{
	colorlinks=true,       % false: boxed links; true: colored links
	linkcolor=red,          % color of internal links (change box color with linkbordercolor)
	citecolor=green,        % color of links to bibliography
	filecolor=magenta,      % color of file links
	urlcolor=cyan           % color of external links
}

\title{Hazard Analysis\\\progname}

\author{\authname}

\date{}

%% Comments

\usepackage{color}

\newif\ifcomments\commentstrue %displays comments
%\newif\ifcomments\commentsfalse %so that comments do not display

\ifcomments
\newcommand{\authornote}[3]{\textcolor{#1}{[#3 ---#2]}}
\newcommand{\todo}[1]{\textcolor{red}{[TODO: #1]}}
\else
\newcommand{\authornote}[3]{}
\newcommand{\todo}[1]{}
\fi

\newcommand{\wss}[1]{\authornote{blue}{SS}{#1}} 
\newcommand{\plt}[1]{\authornote{magenta}{TPLT}{#1}} %For explanation of the template
\newcommand{\an}[1]{\authornote{cyan}{Author}{#1}}


\newcommand{\progname}{Software Eng 4G06} % PUT YOUR PROGRAM NAME HERE
\newcommand{\authname}{Team 2, Parnas' Pals
	\\ Jared Bentvelsen
	\\ Bassel Rezkalla
	\\ Yuvraj Randhawa
	\\ Dimitri Tsampiras
	\\ Matthew McCracken} % AUTHOR NAMES                  

\usepackage{hyperref}
\hypersetup{colorlinks=true, linkcolor=blue, citecolor=blue, filecolor=blue,
	urlcolor=blue, unicode=false}
\urlstyle{same}


\newcounter{ACRnum}
\newcommand{\rtheACRnum}{ACR\theACRnum}
\newcommand{\ACRref}[1]{ACR\ref{#1}}
\newcounter{IRnum} 
\newcommand{\rtheIRnum}{IR\theIRnum}
\newcommand{\IRref}[1]{IR\ref{#1}}
\newcounter{PRRnum}
\newcommand{\rthePRRnum}{PRR\thePRRnum}
\newcommand{\PRRref}[1]{PRR\ref{#1}}
\newcounter{ADRnum}
\newcommand{\rtheADRnum}{ADR\theADRnum}
\newcommand{\ADRref}[1]{ADR\ref{#1}}
\newcounter{IMnum} 
\newcommand{\rtheIMnum}{IM\theIMnum}
\newcommand{\IMref}[1]{IM\ref{#1}}


\begin{document}
	
	\maketitle
	\thispagestyle{empty}
	
	~\newpage
	
	\pagenumbering{roman}
	
	\begin{table}[hp]
		\caption{Revision History} \label{TblRevisionHistory}
		\begin{tabularx}{\textwidth}{llX}
			\toprule
			\textbf{Date} & \textbf{Developer(s)} & \textbf{Change}\\
			\midrule
			Date1 & Name(s) & Description of changes\\
			Date2 & Name(s) & Description of changes\\
			... & ... & ...\\
			\bottomrule
		\end{tabularx}
	\end{table}
	
	~\newpage
	
	\tableofcontents
	
	~\newpage
	
	\pagenumbering{arabic}
	
	\section{Introduction}
	
	The purpose of this document is to identify the components of Olympian and its dependencies that could represent potential risks for Olympian's stakeholders. This document will analyze the risk levels of potentially hazardous components and their associated failures, as well as recommend actions which can be taken to eliminate the resulting risks or mitigate them to an acceptable level.
	
	\subsection{Definition of a Hazard}
	For Olympian's purposes (and throughout this document), a hazard will be defined as any condition or event which can lead to a state that is likely to negatively affect Olympian's stakeholders.
	
	\section{Scope and Purpose of Hazard Analysis}
	
	\section{System Boundaries and Components}
	
	\section{Critical Assumptions}
	
	\wss{These assumptions that are made about the software or system.  You should
		minimize the number of assumptions that remove potential hazards.  For instance,
		you could assume a part will never fail, but it is generally better to include
		this potential failure mode.}
	
	\section{Failure Mode and Effect Analysis}
	\subsection{Hazards Out of Scope}
	\subsection{Failure Modes \& Effects Analysis Table}
	\wss{Include your FMEA table here}
	
	\section{Safety and Security Requirements}
	
	\wss{Newly discovered requirements.  These should also be added to the SRS.  (A
		rationale design process how and why to fake it.)}
	
	\subsection{Access Requirements}
	\noindent 
	\begin{itemize}
		\item[ACR\refstepcounter{ACRnum}\theACRnum:]
		The application must not display other users private details to the user.
	\end{itemize}
	\subsection{Integrity Requirements}
	\noindent 
	\begin{itemize}
		\item[IR\refstepcounter{IRnum}\theIRnum:]
		Passwords must be encrypted with SHA-256 when stored.
	\end{itemize}
	\subsection{Privacy Requirements}
	\noindent 
	\begin{itemize}
		\item[PRR\refstepcounter{PRRnum}\thePRRnum:]
		The application must use OAuth protocols to verify communication between the client and server.
	\end{itemize}
	\subsection{Audit Requirements}
	\noindent
	\begin{itemize}
		\item[ADR\refstepcounter{ADRnum}\theADRnum:]
		Data will be stored in a secure database. When data is deleted or edited a record of this data will be kept for up to 30 days.
	\end{itemize}
	\subsection{Immunity Requirements}
	\noindent 
	\begin{itemize}
		\item N/A
	\end{itemize}
	
	
	\section{Roadmap}
	
	\wss{Which safety requirements will be implemented as part of the capstone timeline?
		Which requirements will be implemented in the future?}
	
	Based on the safety requirements listed above, the table below displays which of the requirements are planned for the current timeline of the project and those that planned implementations for the future.
	
	\begin{center}
		\begin{tabular}{ |c|c| } 
			\hline
			Planned & Future \\
			\hline 
			\hline
			ACR1 & ADR1 \\ 
			\hline
			IR1 & \\
			\hline 
			PRR1 & \\ 
			\hline
		\end{tabular}
	\end{center}
\end{document}