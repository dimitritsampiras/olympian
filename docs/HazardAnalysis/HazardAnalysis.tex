

\documentclass{article}

\usepackage{booktabs}
\usepackage{tabularx}
\usepackage{hyperref}
\usepackage{changepage}
\usepackage{longtable, tabu}

\hypersetup{
	colorlinks=true,       % false: boxed links; true: colored links
	linkcolor=red,          % color of internal links (change box color with linkbordercolor)
	citecolor=green,        % color of links to bibliography
	filecolor=magenta,      % color of file links
	urlcolor=cyan           % color of external links
}

\title{Hazard Analysis\\\progname}

\author{\authname}

\date{}

%% Comments

\usepackage{color}

\newif\ifcomments\commentstrue %displays comments
%\newif\ifcomments\commentsfalse %so that comments do not display

\ifcomments
\newcommand{\authornote}[3]{\textcolor{#1}{[#3 ---#2]}}
\newcommand{\todo}[1]{\textcolor{red}{[TODO: #1]}}
\else
\newcommand{\authornote}[3]{}
\newcommand{\todo}[1]{}
\fi

\newcommand{\wss}[1]{\authornote{blue}{SS}{#1}} 
\newcommand{\plt}[1]{\authornote{magenta}{TPLT}{#1}} %For explanation of the template
\newcommand{\an}[1]{\authornote{cyan}{Author}{#1}}


\newcommand{\progname}{Software Eng 4G06} % PUT YOUR PROGRAM NAME HERE
\newcommand{\authname}{Team 2, Parnas' Pals
	\\ Jared Bentvelsen
	\\ Bassel Rezkalla
	\\ Yuvraj Randhawa
	\\ Dimitri Tsampiras
	\\ Matthew McCracken} % AUTHOR NAMES                  

\usepackage{hyperref}
\hypersetup{colorlinks=true, linkcolor=blue, citecolor=blue, filecolor=blue,
	urlcolor=blue, unicode=false}
\urlstyle{same}


\newcounter{ACRnum}
\newcommand{\rtheACRnum}{ACR\theACRnum}
\newcommand{\ACRref}[1]{ACR\ref{#1}}
\newcounter{IRnum} 
\newcommand{\rtheIRnum}{IR\theIRnum}
\newcommand{\IRref}[1]{IR\ref{#1}}
\newcounter{PRRnum}
\newcommand{\rthePRRnum}{PRR\thePRRnum}
\newcommand{\PRRref}[1]{PRR\ref{#1}}
\newcounter{ADRnum}
\newcommand{\rtheADRnum}{ADR\theADRnum}
\newcommand{\ADRref}[1]{ADR\ref{#1}}
\newcounter{IMnum} 
\newcommand{\rtheIMnum}{IM\theIMnum}
\newcommand{\IMref}[1]{IM\ref{#1}}


\begin{document}
	
	\maketitle
	\thispagestyle{empty}
	
	~\newpage
	
	\pagenumbering{roman}

	\tableofcontents

	\newpage

	\section{Revision History}
	
	\begin{table}[hp]
		\caption{Revision History} \label{TblRevisionHistory}
		\begin{tabularx}{\textwidth}{llX}
			\toprule
			\textbf{Date} & \textbf{Developer(s)} & \textbf{Change}\\
			\midrule
			19/10/22 & All & Initial Draft\\
			\bottomrule
		\end{tabularx}
	\end{table}
	
	~\newpage
	
	\pagenumbering{arabic}
	
	\section{Introduction}
	
	The purpose of this document is to identify the components of Olympian and its dependencies that could represent potential risks for Olympian's stakeholders. This document will analyze the risk levels of potentially hazardous components and their associated failures, as well as recommend actions which can be taken to eliminate the resulting risks or mitigate them to an acceptable level.
	
	\subsection{Definition of a Hazard}
	For Olympian's purposes (and throughout this document), a hazard will be defined as any condition or event which can lead to a state that is likely to negatively affect Olympian's stakeholders.
	
	\section{Scope and Purpose of Hazard Analysis}

	This document aims to provide an in-depth analysis to potential system hazards of the
	Olympian app. These hazards encompass categories including security, authorization,
	input correctness and error handling.

	\section{System Boundaries and Components}

	The system upon which the Hazard Analysis will be performed on consists of the following components:

	\begin{enumerate}
		\item The Olympian mobile application, which is composed of a front-end interface served by a back-end server, supports the following major functionalities: 
		\begin{enumerate}
			\item Profile Creation
			\item Workout Routine Creation
			\item Workout Routine Discovery and Browsing
			\item Workout Routine Reviewing
			\item Workout Progress Tracking
			\item Long Term Goal Progress Tracking
		\end{enumerate}

		\item The physical Android or iOS mobile device.

		\item The AWS Redshift Database where relational user information is stored.

	\end{enumerate}
	Although integral to the system, the physical mobile device and Redshift Database availability are not under the control of Parnas' Pals.
    The physical mobile device is manufactured by a third party company, and operated by the user. 
	The Redshift Database is operated by Amazon Web Services, making them responsible for database availability.
	
	\section{Critical Assumptions}

	\begin{itemize}

	\item The user is assumed to have the Olympian application downloaded.

	\item The user's mobile device is assumed to have internet access.

	\item The user is assumed to have basic mobile device skills such as tapping the screen and swiping.
		
	\end{itemize}

	\section{Failure Mode and Effect Analysis}
	\subsection{Hazards Out of Scope}
	\begin{enumerate}
		\item Native Mobile Device Software Failures: Because this is a mobile application, many native features will be used to provide application functionality.
		These features can include haptic feedback, notifications, accessibility controls, etc. It is possible that some of these
		features may fail on the mobile device, which would create a hazard for the application, but is outside the control of the application developers. Additionally, the application
		will only function on iOS 10 / Android 5.0 and above (React Native is unsupported). The version of software used by the user mobile device is not within the control of the application developers, and therefore
		the application will not be available to devices with operating system software older than iOS 10 or Android 5.0.
		
		\item Database and Cloud Hosting Service Failures: The application relies on external services such as AWS Redshift to store and retrieve data, and process user requests.
		The availability of these services is not under the control of the developers of the application, and interruptions in their availability presents a hazard for the application.
	\end{enumerate}
	These hazards cannot be prevented by the application developers but will be mitigated to the fullest possible extent.
	\subsection{Failure Modes \& Effects Analysis Table}
	\begin{adjustwidth}{-3cm}{-1cm}
	\centering
	\noindent\begin{tabular}{ |p{3.3cm}|p{2.5cm}|p{3.2cm}|p{2cm}|p{2.5cm}|p{1cm}|p{1cm}| @{} }
		\hline
		Component & Failure Modes & Effects Of Failure & Causes Of Failure & Recommended Action & SR & Ref.\\
		\hline
		User Login \& Authentication& User cannot log in to application & User cannot utilize site functionality & 
			a. User uses incorrect login credentials.
			& 
			Reset user credentials.
			& PRR1 & H1-1\\
		\hline
		User Private Data Access & Data that is meant to be kept private is displayed publicly & 
			User privacy is breached and sensitive data is released & 
				a. User privacy settings are incorrectly stored.
				\newline
				b. Malicious third party gains access to user data.
			& 
				User data will be backed up daily to avoid error. Utilize stringent AWS admin permissions.
			& ACR1, ACR2, IR1, ADR1 & H2-1\\
		\hline
		Workout Suggestion Algorithm& Workouts are incorrectly or illegally accessed by users& Users are able to access unavailable, restricted or un-catered routines & a. Database failure \newline b. Privacy system failure \newline c. Suggestion algorithm failure & Display detailed message to user on attempt to access restricted routine & PRR1 &H3-1\\
		\hline
		Application Server& Application Server terminates unexpectedly & Current data transactions and communication will cease& a. Host failure \newline b. server exceeds data limit& Communicate server issues to users and store unsaved data locally on the user device & IR3, IR4, IR5 & H4-1 \\
		\hline

 	\end{tabular}

	 \noindent\begin{tabular}{ |p{3.3cm}|p{2.5cm}|p{3.2cm}|p{2cm}|p{2.5cm}|p{1cm}|p{1cm}| @{} }
	\hline
	Database& Data is deleted unintentionally& Collected data will not be available for user display or system analysis& a. Database failure & Regularly and automatically backup database and allow admin permissions to rollback & IR5, ADR1& H5-1\\
	\cline{2-7}
	& Database is unavailable & Data transactions will be unavailable & a. Database failure \newline b. Host failure& Refer to H5-1 & IR3, IR4 & H5-2\\
	\cline{2-7}
	& Required data is not accessible &  Data transactions will be unavailable to certain users. & a. Database failure \newline b. Host failure & Refer to H5-1 & IR3, IR4 &H5-3\\
	\hline
	User Interface & Components and component data does not successfully render onto to UI component & User will miss vital cues, prompts, and information required to operate
	the app & Native libraries, components, and properties used on incompatible operating systems & The system should accommodate various operating systems and
	their versions. Animated components using native drivers need to be disabled on devices and OS versions that do not support them. 
	The system should frequently check which OS platform the app is running on to ensure the use of OS specific UI elements. & PRR2, PRR3 & H6-1\\
	\hline
	\end{tabular}
	\end{adjustwidth}
	\section{Safety and Security Requirements}
	
	\subsection{Access Requirements}
	\noindent 
	\begin{itemize}
		\item[ACR\refstepcounter{ACRnum}\theACRnum:]
		The application must not display other users private details to the user.
		\item[ACR\refstepcounter{ACRnum}\theACRnum:]
		Only the developers and system administrators will be able to access all user details except their passwords.
	\end{itemize}
	\subsection{Integrity Requirements}
	\noindent 
	\begin{itemize}
		\item[IR\refstepcounter{IRnum}\theIRnum:]
		Passwords must be encrypted with SHA-256 when stored.
		\item[IR\refstepcounter{IRnum}\theIRnum:]
		User data will not be modified void of user permission.
		\item[IR\refstepcounter{IRnum}\theIRnum:]
		User data will be automatically backed up to the database upon connection to the internet.
		\item[IR\refstepcounter{IRnum}\theIRnum:]
		User data transactions will be stored locally when user device is offline.
		\item[IR\refstepcounter{IRnum}\theIRnum:]
		Database will be backed up daily.
	\end{itemize}
	\subsection{Privacy Requirements}
	\noindent 
	\begin{itemize}
		\item[PRR\refstepcounter{PRRnum}\thePRRnum:]
		The application must use OAuth protocols to verify communication between the client and server.
		\item[PRR\refstepcounter{PRRnum}\thePRRnum:]
		The application will ensure users are aware of data collection practices before collecting any data from them.
		\item[PRR\refstepcounter{PRRnum}\thePRRnum:]
		The application will communicate any changes to the privacy policy with the users.	
	\end{itemize}
	\subsection{Audit Requirements}
	\noindent
	\begin{itemize}
		\item[ADR\refstepcounter{ADRnum}\theADRnum:]
		Data will be stored in a secure database. When data is deleted or edited a record of this data will be kept for up to 30 days.
	\end{itemize}
	\subsection{Immunity Requirements}
	\noindent 
	\begin{itemize}
		\item N/A
	\end{itemize}
	\newpage
	
	\section{Roadmap}
	
	
	Based on the safety requirements listed above, the table below displays which of the requirements are planned for the current timeline of the project and those that planned implementations for the future.
	
	\begin{center}
		\begin{tabular}{ |c|c| } 
			\hline
			Planned & Future \\
			\hline 
			\hline
			ACR1 & ADR1 \\ 
			\hline
			ACR2 & \\
			\hline
			IR1 & \\
			\hline
			IR2 & \\
			\hline
			IR3 & \\
			\hline
			IR4 & \\
			\hline
			IR5 & \\
			\hline 
			PRR1 & \\ 
			\hline
			PR2 & \\
			\hline
			PR3 & \\
			\hline
		\end{tabular}
	\end{center}
\end{document}