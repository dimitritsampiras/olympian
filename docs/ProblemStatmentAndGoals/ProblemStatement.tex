\documentclass{article}

\usepackage{tabularx}
\usepackage{booktabs}

\title{Problem Statement and Goals\\\progname}

\author{\authname}

\date{}

%% Comments

\usepackage{color}

\newif\ifcomments\commentstrue %displays comments
%\newif\ifcomments\commentsfalse %so that comments do not display

\ifcomments
\newcommand{\authornote}[3]{\textcolor{#1}{[#3 ---#2]}}
\newcommand{\todo}[1]{\textcolor{red}{[TODO: #1]}}
\else
\newcommand{\authornote}[3]{}
\newcommand{\todo}[1]{}
\fi

\newcommand{\wss}[1]{\authornote{blue}{SS}{#1}} 
\newcommand{\plt}[1]{\authornote{magenta}{TPLT}{#1}} %For explanation of the template
\newcommand{\an}[1]{\authornote{cyan}{Author}{#1}}


\newcommand{\progname}{Software Eng 4G06} % PUT YOUR PROGRAM NAME HERE
\newcommand{\authname}{Team 2, Parnas' Pals
	\\ Jared Bentvelsen
	\\ Bassel Rezkalla
	\\ Yuvraj Randhawa
	\\ Dimitri Tsampiras
	\\ Matthew McCracken} % AUTHOR NAMES                  

\usepackage{hyperref}
\hypersetup{colorlinks=true, linkcolor=blue, citecolor=blue, filecolor=blue,
	urlcolor=blue, unicode=false}
\urlstyle{same}



\begin{document}

\maketitle

\begin{table}[hp]
\caption{Revision History} \label{TblRevisionHistory}
\begin{tabularx}{\textwidth}{llX}
\toprule
\textbf{Date} & \textbf{Developer(s)} & \textbf{Change}\\
\midrule
Date1 & Name(s) & Description of changes\\
Date2 & Name(s) & Description of changes\\
... & ... & ...\\
\bottomrule
\end{tabularx}
\end{table}

\section{Problem Statement}

\wss{You should check your problem statement with the
\href{https://github.com/smiths/capTemplate/blob/main/docs/Checklists/ProbState-Checklist.pdf}
{problem statement checklist}.}
\wss{You can change the section headings, as long as you include the required information.}

\subsection{Problem}

\subsection{Inputs and Outputs}

\wss{Characterize the problem in terms of ``high level'' inputs and outputs.  
Use abstraction so that you can avoid details.}

\subsection{Stakeholders}

\subsection{Environment}

\wss{Hardware and software}

\section{Goals}

	\begin{tabular}{ |p{5cm}|p{8cm}| }
	\hline
	\multicolumn{2}{|c|}{Goal Descriptions} \\
	\hline
	Goal Name & blah blah blah \\
	\hline
	Name & blah blah blah \\
	\hline
	Name & blah blah blah \\
	\hline
	Name & blah blah blah \\
	\hline
	Name & blah blah blah \\
	\hline
\end{tabular}

\subsection{Stretch Goals}

\section{Inputs and Outputs}

\section {Development Plan}

\subsection{Team meeting plan}
Team meetings will be conducted weekly on non-holiday Mondays at 4:30pm.
Further meetings can be scheduled if needed.
Meetings will be conducted in person at a McMaster bookable room or library unless specified otherwise.

\subsection{Team communication plan}
Common team communication and questions will be conducted on \textbf{Mircosoft Teams}.\\
Any online meetings will be conducted on \textbf{Mircosoft Teams}. \\

\subsection{Workflow plan}
GitHub will be used for all workflow and version control.

\subsubsection{Task Scheduling}
The \textbf{Github Kanban} project board will be used to plan and assign upcomming and ongoing tasks to team developers. \\
If a task is large or general, sub-tasks will be created and assigend accordingly.\\
Upcomming project delieverables must include deadlines.\\

\subsubsection{Git Branch Usage}
All features, documents and file changes require a named branch describing the change.
Branches will follow a similar format to:

\begin{center}
	$topic\_or\_type/section/branch\_description$
\end{center}

\subsubsection{Git Commits}
\textbf{Tags} will be attached to commits as needed to further display the change description.\\
Commit Squashing may be used prior to creating pull requests to clean up unnecessary commits.

\subsubsection{Git Branch Merging and Pull Requests}
The main or centered branch will be protected such that new features and addtions will require a pull request in order to be merged. \\
Pull Requests require a minimum of 2 approvals to be accepted.

\subsubsection{Technial Issues}
For all technial questions, concerns and issues, the \textbf{Github Issues} feature will be used to open an issue or concern accordingly.

\subsection{Team Roles}
\begin{tabular}{ |p{5cm}|p{8cm}| }
	\hline
	\multicolumn{2}{|c|}{Team Role Descriptions} \\
	\hline
	Role & blah blah blah \\
	\hline
	Front End Expert & Dimitri \\
	\hline
	Database Expert  & Jared \\
	\hline
	UX/UI Expert & Matt \\
	\hline
	Role & blah blah blah \\
	\hline
\end{tabular}

\subsection{Coding Style and Standards}
The \textbf{Google Coding Style} will be used as a standard for all code in the project. \\
The style guide for this can be found \hyperlink{https://google.github.io/styleguide/}{here.}
A language specific linter will be enforced.

\subsection{Testing}


\subsection {Risks}


\end{document}