\documentclass{article}

\usepackage{tabularx}
\usepackage{booktabs}

\title{Problem Statement and Goals\\\progname}

\author{\authname}

\date{}

%% Comments

\usepackage{color}

\newif\ifcomments\commentstrue %displays comments
%\newif\ifcomments\commentsfalse %so that comments do not display

\ifcomments
\newcommand{\authornote}[3]{\textcolor{#1}{[#3 ---#2]}}
\newcommand{\todo}[1]{\textcolor{red}{[TODO: #1]}}
\else
\newcommand{\authornote}[3]{}
\newcommand{\todo}[1]{}
\fi

\newcommand{\wss}[1]{\authornote{blue}{SS}{#1}} 
\newcommand{\plt}[1]{\authornote{magenta}{TPLT}{#1}} %For explanation of the template
\newcommand{\an}[1]{\authornote{cyan}{Author}{#1}}


\newcommand{\progname}{Software Eng 4G06} % PUT YOUR PROGRAM NAME HERE
\newcommand{\authname}{Team 2, Parnas' Pals
	\\ Jared Bentvelsen
	\\ Bassel Rezkalla
	\\ Yuvraj Randhawa
	\\ Dimitri Tsampiras
	\\ Matthew McCracken} % AUTHOR NAMES                  

\usepackage{hyperref}
\hypersetup{colorlinks=true, linkcolor=blue, citecolor=blue, filecolor=blue,
	urlcolor=blue, unicode=false}
\urlstyle{same}



\begin{document}

\maketitle

\begin{table}[hp]
\caption{Revision History} \label{TblRevisionHistory}
\begin{tabularx}{\textwidth}{llX}
\toprule
\textbf{Date} & \textbf{Developer(s)} & \textbf{Change}\\
\midrule
Date1 & Name(s) & Description of changes\\
Date2 & Name(s) & Description of changes\\
... & ... & ...\\
\bottomrule
\end{tabularx}
\end{table}

\section{Problem Statement}

\wss{You should check your problem statement with the
\href{https://github.com/smiths/capTemplate/blob/main/docs/Checklists/ProbState-Checklist.pdf}
{problem statement checklist}.}
\wss{You can change the section headings, as long as you include the required information.}

\subsection{Problem}

\subsection{Inputs and Outputs}

\wss{Characterize the problem in terms of ``high level'' inputs and outputs.  
Use abstraction so that you can avoid details.}

\subsection{Stakeholders}

\subsection{Environment}

\wss{Hardware and software}

\section{Goals}

	\begin{tabular}{ |p{5cm}|p{8cm}| }
	\hline
	\multicolumn{2}{|c|}{Goal Descriptions} \\
	\hline
	Goal Name & Explanation \\
	\hline
	The app should provide a fast, efficient, and easy way to create Programs. & Creating a program shouldn’t feel like pulling teeth. It should feel more intuitive and convinient than creating a program on an excel spreadsheet. Otherwise the value of using this app decreases.  \\
	\hline
	App provides and accessible way for users to search for programs that are inline with their goals. & blah blah blah \\
	\hline
	User’s can use active programs in a way that doesn’t intrude or interupt their workout & Tracking/logging workouts is an essential part to ensuring that the user is on track with their goals. However, it can be a tedious process that is often left out. The success of the app will be heavily depedant on whether tracking and logging workouts isn’t a tedious process. \\
	\hline
	Programs should be polymorphic - they can accomodate any style of workout. & Fitness programs can be pretty diverse. There’s a wide range of way’s to configure a program. These options include duration, workout intervals, exercise content, etc. It is important for the program creation section of the app to be efficient in creating a diverse range of programs. \\
  \hline
	App presents data about the user’s active program that is insightful and analytical. & Not seeing immediate results can be turn people away from working around \\
	\hline
\end{tabular}

\subsection{Stretch Goals}

\section{Inputs and Outputs}

\section {Development Plan}

\subsection{Team meeting plan}
Team meetings will be conducted weekly on non-holiday Mondays at 4:30pm.
Further meetings can be scheduled if needed.
Meetings will be conducted in person at a McMaster bookable room or library unless specified otherwise.

\subsection{Team communication plan}
All team communication will be conducted 
Any online meetings will be conducted on Mircosoft teams in a specified 

\subsection{Workflow plan}
git

\subsection{Team Roles}
\begin{tabular}{ |p{5cm}|p{8cm}| }
	\hline
	\multicolumn{2}{|c|}{Team Role Descriptions} \\
	\hline
	Role & blah blah blah \\
	\hline
	Role & blah blah blah \\
	\hline
	Role & blah blah blah \\
	\hline
	Role & blah blah blah \\
	\hline
	Role & blah blah blah \\
	\hline
\end{tabular}

\subsection{Coding Style and Standards}
lint

\section {POC Demo Plan}

\end{document}