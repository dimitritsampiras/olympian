\documentclass{article}

\usepackage{tabularx}
\usepackage{booktabs}
\setlength\parindent{0pt}
\title{Problem Statement and Goals\\\progname}

\author{\authname}

\date{}

%% Comments

\usepackage{color}

\newif\ifcomments\commentstrue %displays comments
%\newif\ifcomments\commentsfalse %so that comments do not display

\ifcomments
\newcommand{\authornote}[3]{\textcolor{#1}{[#3 ---#2]}}
\newcommand{\todo}[1]{\textcolor{red}{[TODO: #1]}}
\else
\newcommand{\authornote}[3]{}
\newcommand{\todo}[1]{}
\fi

\newcommand{\wss}[1]{\authornote{blue}{SS}{#1}} 
\newcommand{\plt}[1]{\authornote{magenta}{TPLT}{#1}} %For explanation of the template
\newcommand{\an}[1]{\authornote{cyan}{Author}{#1}}


\newcommand{\progname}{Software Eng 4G06} % PUT YOUR PROGRAM NAME HERE
\newcommand{\authname}{Team 2, Parnas' Pals
	\\ Jared Bentvelsen
	\\ Bassel Rezkalla
	\\ Yuvraj Randhawa
	\\ Dimitri Tsampiras
	\\ Matthew McCracken} % AUTHOR NAMES                  

\usepackage{hyperref}
\hypersetup{colorlinks=true, linkcolor=blue, citecolor=blue, filecolor=blue,
	urlcolor=blue, unicode=false}
\urlstyle{same}



\begin{document}

\maketitle

\begin{table}[hp]
\caption{Revision History} \label{TblRevisionHistory}
\begin{tabularx}{\textwidth}{llX}
\toprule
\textbf{Date} & \textbf{Developer(s)} & \textbf{Change}\\
\midrule
24/09/22 & Jared Bentvelsen, Yuvraj Randhawa, Bassel Rezkalla & Initial draft of Problem Statement\\
Date2 & Name(s) & Description of changes\\
... & ... & ...\\
\bottomrule
\end{tabularx}
\end{table}

\section{Problem Statement}

\subsection{Problem}

Working out and exercising is a vital component of any healthy lifestyle, and has been shown to greatly improve both physical and mental well-being. However,
many individuals feel reluctant to go to a gym or establish any consistent workout routines simply because they aren't sure what to do. 
Existing exercise content is mostly expensive Excel spreadsheets or brief, hard-to-follow Tik-Tok clips which are inconvenient to use while you're exercising or in the gym.
There does not yet exist an application to track progress and discover new workout content in an easy and effective manner.
Workout routines can vary infinitely depending on an athlete's goals and preferences. A workout can be composed of any number of exercises
and each of these exercises can be done in any number of variations, with different technical adjustments. 
\\
The high level goal of this project is to
make working out highly accessible for beginners and to spread new tips and techniques to even the most experienced athletes in a clean and accessible manner.

\subsection{Inputs and Outputs}

\textbf{Inputs}: 
\begin{enumerate}
	\item A user's favourite workout routines for accomplishing a specific goal (e.g. build strength on bench press) for others to discover
	\begin{enumerate}
		\item General workout description
		\item A set of exercises
		\item Sets, reps, and weight where applicable
		\item Technical comments, pieces of advice
		\item Self recorded technique demonstration videos
	\end{enumerate}
	\item Comments and reviews for other posted workout routines
	\item Fitness goals that a user wishes to reach in an optional time frame (e.g. Run 5km in under 25 minutes, or lift 225lbs on bench press)
	\item A user's actions during a workout, or data progressing towards a set goal (e.g. Record weight on the bench press, and `tick off' exercises as they are performed during a session.)
\end{enumerate}

\textbf{Outputs}:
\begin{enumerate}
	\item Searchable workout content for users to discover programs for specific goals.
	\item A method for users to seamlessly track their own progress through a routine or towards their more general goals.
	\item Reviews and comments left by other users indicating the quality of a given workout routine.
\end{enumerate}

% \wss{Characterize the problem in terms of ``high level'' inputs and outputs.  
% Use abstraction so that you can avoid details.}

\subsection{Stakeholders}
\begin{enumerate}
	\item Anyone interested in exercising (both beginner and advanced)
	\item Personal trainers
	\item Fitness Advertisers
\end{enumerate}

\subsection{Environment}
Supported platforms: iOS/Android mobile application, web application accessible from browser

% \wss{Hardware and software}

\section{Goals}

	\begin{tabular}{ |p{5cm}|p{8cm}| }
	\hline
	\multicolumn{2}{|c|}{Goal Descriptions} \\
	\hline
	Goal Name & blah blah blah \\
	\hline
	Name & blah blah blah \\
	\hline
	Name & blah blah blah \\
	\hline
	Name & blah blah blah \\
	\hline
	Name & blah blah blah \\
	\hline
\end{tabular}

\subsection{Stretch Goals}

\section{Inputs and Outputs}

\section {Development Plan}

\subsection{Team meeting plan}
Team meetings will be conducted weekly on non-holiday Mondays at 4:30pm.
Further meetings can be scheduled if needed.
Meetings will be conducted in person at a McMaster bookable room or library unless specified otherwise.

\subsection{Team communication plan}
Common team communication and questions will be conducted on \textbf{Mircosoft Teams}.\\
Any online meetings will be conducted on \textbf{Mircosoft Teams}. \\

\subsection{Workflow plan}
GitHub will be used for all workflow and version control.

\subsubsection{Task Scheduling}
The \textbf{Github Kanban} project board will be used to plan and assign upcomming and ongoing tasks to team developers. \\
If a task is large or general, sub-tasks will be created and assigend accordingly.\\
Upcomming project delieverables must include deadlines.\\

\subsubsection{Git Branch Usage}
All features, documents and file changes require a named branch describing the change.
Branches will follow a similar format to:

\begin{center}
	$topic\_or\_type/section/branch\_description$
\end{center}

\subsubsection{Git Commits}
\textbf{Tags} will be attached to commits as needed to further display the change description.\\
Commit Squashing may be used prior to creating pull requests to clean up unnecessary commits.

\subsubsection{Git Branch Merging and Pull Requests}
The main or centered branch will be protected such that new features and addtions will require a pull request in order to be merged. \\
Pull Requests require a minimum of 2 approvals to be accepted.

\subsubsection{Technial Issues}
For all technial questions, concerns and issues, the \textbf{Github Issues} feature will be used to open an issue or concern accordingly.

\subsection{Team Roles}
\begin{tabular}{ |p{5cm}|p{8cm}| }
	\hline
	\multicolumn{2}{|c|}{Team Role Descriptions} \\
	\hline
	Role & blah blah blah \\
	\hline
	Front End Expert & Dimitri \\
	\hline
	Database Expert  & Jared \\
	\hline
	UX/UI Expert & Matt \\
	\hline
	Role & blah blah blah \\
	\hline
\end{tabular}

\subsection{Coding Style and Standards}
The \textbf{Google Coding Style} will be used as a standard for all code in the project. \\
The style guide for this can be found \hyperlink{https://google.github.io/styleguide/}{here.}
A language specific linter will be enforced.

\subsection{Testing}

\subsubsection {Unit Testing}
\textbf{\href{https://jestjs.io/}{Jest}} will be used for this app's unit testing. Jest offers mocking and code coverage insights which will be useful for unit testing.

\subsubsection {Simulation / Integration Testing}
\textbf{\href{https://testcafe.io/}{TestCafe}}} will be used for this app's integration testing. TestCafe is a cross-browser integration testing framework that will be used to verify proper integration between the app's various components (back-end, front-end, database, etc). TestCafe also offers end-to-end testing which includes automatic simulation of many common web-app components such as forms and buttons.

\subsection {Risks}


\end{document}