\documentclass{article}

\usepackage{tabularx}
\usepackage{booktabs}
\setlength\parindent{0pt}
\title{Problem Statement and Goals\\\progname}

\author{\authname}

\date{}

%% Comments

\usepackage{color}

\newif\ifcomments\commentstrue %displays comments
%\newif\ifcomments\commentsfalse %so that comments do not display

\ifcomments
\newcommand{\authornote}[3]{\textcolor{#1}{[#3 ---#2]}}
\newcommand{\todo}[1]{\textcolor{red}{[TODO: #1]}}
\else
\newcommand{\authornote}[3]{}
\newcommand{\todo}[1]{}
\fi

\newcommand{\wss}[1]{\authornote{blue}{SS}{#1}} 
\newcommand{\plt}[1]{\authornote{magenta}{TPLT}{#1}} %For explanation of the template
\newcommand{\an}[1]{\authornote{cyan}{Author}{#1}}


\newcommand{\progname}{Software Eng 4G06} % PUT YOUR PROGRAM NAME HERE
\newcommand{\authname}{Team 2, Parnas' Pals
	\\ Jared Bentvelsen
	\\ Bassel Rezkalla
	\\ Yuvraj Randhawa
	\\ Dimitri Tsampiras
	\\ Matthew McCracken} % AUTHOR NAMES                  

\usepackage{hyperref}
\hypersetup{colorlinks=true, linkcolor=blue, citecolor=blue, filecolor=blue,
	urlcolor=blue, unicode=false}
\urlstyle{same}



\begin{document}

\maketitle

\begin{table}[hp]
\caption{Revision History} \label{TblRevisionHistory}
\begin{tabularx}{\textwidth}{llX}
\toprule
\textbf{Date} & \textbf{Developer(s)} & \textbf{Change}\\
\midrule
24/09/22 & Jared Bentvelsen, Yuvraj Randhawa, Bassel Rezkalla & Initial draft of Problem Statement\\
Date2 & Name(s) & Description of changes\\
... & ... & ...\\
\bottomrule
\end{tabularx}
\end{table}

\section{Problem Statement}

\subsection{Problem}

Working out and exercising is a vital component of any healthy lifestyle, and has been shown to greatly improve both physical and mental well-being. However,
many individuals feel reluctant to go to a gym or establish any consistent workout routines simply because they aren't sure what to do. 
Existing exercise content is mostly expensive Excel spreadsheets or brief, hard-to-follow Tik-Tok clips which are inconvenient to use while you're exercising or in the gym.
There does not yet exist an application to track progress and discover new workout content in an easy and effective manner.
Workout routines can vary infinitely depending on an athlete's goals and preferences. A workout can be composed of any number of exercises
and each of these exercises can be done in any number of variations, with different technical adjustments. 
\\
The high level goal of this project is to
make working out highly accessible for beginners and to spread new tips and techniques to even the most experienced athletes in a clean and accessible manner.

\subsection{Inputs and Outputs}

\textbf{Inputs}: 
\begin{enumerate}
	\item A user's favourite workout routines for accomplishing a specific goal (e.g. build strength on bench press) for others to discover
	\begin{enumerate}
		\item General workout description
		\item A set of exercises
		\item Sets, reps, and weight where applicable
		\item Technical comments, pieces of advice
		\item Self recorded technique demonstration videos
	\end{enumerate}
	\item Comments and reviews for other posted workout routines
	\item Fitness goals that a user wishes to reach in an optional time frame (e.g. Run 5km in under 25 minutes, or lift 225lbs on bench press)
	\item A user's actions during a workout, or data progressing towards a set goal (e.g. Record weight on the bench press, and `tick off' exercises as they are performed during a session.)
\end{enumerate}

\textbf{Outputs}:
\begin{enumerate}
	\item Searchable workout content for users to discover programs for specific goals.
	\item A method for users to seamlessly track their own progress through a routine or towards their more general goals.
	\item Reviews and comments left by other users indicating the quality of a given workout routine.
\end{enumerate}

% \wss{Characterize the problem in terms of ``high level'' inputs and outputs.  
% Use abstraction so that you can avoid details.}

\subsection{Stakeholders}
\begin{enumerate}
	\item Anyone interested in exercising (both beginner and advanced)
	\item Personal trainers
	\item Fitness Advertisers
\end{enumerate}

\subsection{Environment}
Supported platforms: iOS/Android mobile application, web application accessible from browser

% \wss{Hardware and software}

\section{Goals}

\begin{tabular}{ |p{5cm}|p{8cm}| }
	\hline
	Goal Name & Explanation \\
	\hline
	The app should provide a fast, efficient, and easy way to create programs. & Creating a program shouldn't feel like pulling teeth. It should feel more intuitive and convenient than creating a program on an excel spreadsheet - otherwise the value provided by this app decreases.  \\
	\hline
	App provides an accessible way for users to discover programs that are in-line with their goals & A user should be able to search for programs in a simple manner. The programs presented should be very informative in regards to whether or not they work. \\
	\hline
	Users can use active programs in a way that doesn't intrude or interrupt their workout & Tracking/logging workouts is an essential part to ensuring that the user is on track with their goals. However, it has traditionally been a tedious process that is often neglected. Tracking and logging workouts should be a streamlined process. \\
	\hline
	Programs should be polymorphic - they can accommodate any style of workout. & Fitness programs can be quite diverse. There’s a wide range of ways to configure a program - programs can vary by duration, workout intervals, exercise content, etc. It is important for the program creation process to allow efficient creation of diverse ranges of programs. \\
  \hline
	App presents data about the user's active program that is insightful and analytical that may not be evident to the user. & Not seeing immediate results can turn people away from sticking with their workout regiment. Displaying insightful user statistics throughout the duration of their program can create a positive feedback loop of hard work vs. rewards. It's important that the statistics presented are not necessarily obvious and evident to the user.   \\
	\hline
\end{tabular}

\subsection{Stretch Goals}

\begin{tabular}{ |p{5cm}|p{8cm}| }
	\hline
	Goals & Explanation \\
	\hline
	Diet Planning & Users can track their diets and include diet information in their custom workout plans. Some workouts are
	designed around bulking or cutting calories and so including diet info will improve utility. \\
	\hline
	Recommended Workouts Algorithm & Users can receive new workouts in a "Recommended" tab. These workouts will be selected for the user by analyzing
	their past plan history and plan tags. Popular and new workouts will be recommended to the user. This will improve discoverability and improve
	 user experience and keep them on the app for longer.  \\
	\hline
	History of growth and projection of potential future growth & tbd \\
	\hline
	Social Growth Analytics & Users can view a network of other users who have similar goals and learn from their progress.
	Users who improved their abilities by following a regimen can act as a guide to others looking to make progress. This feature can improve
	users' ability to see success with our app. \\
	\hline
\end{tabular}

\section {Development Plan}

\subsection{Team meeting plan}
Team meetings will be conducted weekly on non-holiday Mondays at 4:30pm.
Further meetings can be scheduled if needed.
Meetings will be conducted in person at a McMaster bookable room or library unless specified otherwise.

\subsection{Team communication plan}
Common team communication and questions will be conducted on \textbf{Microsoft Teams}.\\
Any online meetings will be conducted on \textbf{Microsoft Teams}. \\

\subsection{Workflow plan}
GitHub will be used for all workflow and version control.

\subsubsection{Task Scheduling}
The \textbf{Github Kanban} project board will be used to plan and assign upcoming and ongoing tasks to team developers. \\
If a task is large or general, sub-tasks will be created and assigned accordingly.\\
Upcoming project deliverables must include deadlines.\\

\subsubsection{Git Branch Usage}
All features, documents and file changes require a named branch describing the change.
Branches will follow a similar format to:

\begin{center}
	$topic\_or\_type/section/branch\_description$
\end{center}

\subsubsection{Git Commits}
\textbf{Tags} will be attached to commits as needed to further display the change description.\\
Commit Squashing may be used prior to creating pull requests to clean up unnecessary commits.

\subsubsection{Git Branch Merging and Pull Requests}
The main or centered branch will be protected such that new features and additions will require a pull request in order to be merged. \\
Pull Requests require a minimum of 2 approvals to be accepted.

\subsubsection{Technial Issues}
For all technical questions, concerns and issues, the \textbf{Github Issues} feature will be used to open an issue or concern accordingly.

\subsection{Team Roles}
\begin{tabular}{ |p{5cm}|p{8cm}| }
	\hline
	\multicolumn{2}{|c|}{Team Role Descriptions} \\
	\hline
	Meeting Organizer & William \\
	\hline
	Front End Expert & Dimitri \\
	\hline
	Database Experts  & Jared, Bassel \\
	\hline
	UX/UI Expert & Matthew, Dimitri \\
	\hline
	Networking Protocols Expert & Bassel \\
	\hline
	Backend Expert & Jared, William \\
	\hline
	Lead Testers & Yuvi, Bassel \\
	\hline
	Team Liason & Jared \\
	\hline
	Backup-Team Liason & William \\
	\hline
	Git Experts & Yuvi, Matthew\\
	\hline
	Scrum Leader & William \\
	\hline
	Mobile Development Expert & Dimitri, Matthew \\
	\hline
	Latex Expert & Yuvi \\
	\hline
	Hardware Expert & Bassel \\
	\hline
\end{tabular}

\subsdaksml

Garbanzooooo

thise is a chgne anotr oper tomat diff make a gnaege anote benw

\end{document}